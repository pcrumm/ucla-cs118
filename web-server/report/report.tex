\title{Concurrent Web Server Using BSD Sockets}
\author{
	Phil Crumm \\
	804-005-575
		\and
	Ivan Petkov \\
	704-046-431
}
\date{\today}

\documentclass[12pt]{article}

\begin{document}
\maketitle

\section{Web Server Design}
We have created a very rudimentary web server using BSD sockets. Upon starting the server, we create one socket using system-level calls. This socket is created, opened, bound to a port, begins listening, and is finally told to accept connections on another socket. Following this, we pass the received request off to a handler function. This handler function parses the request to determine which filename is requested (by tokenizing the request by spaces and looking for the 1-position element). It then looks for the requested file on the filesystem, and responds appropriately.

By default, all requests are served with a MIME type of text/plain. We check the file extension to recognize other cases: if a file is suffixed with \emph{.html}, the file is served as text/html. Similarly, if the file is suffixed with \emph{.jpg} or \emph{.gif}, the appropriate image MIME type will be used.

\section{Implementation Difficulties}

\section{Running The Server}
To run the server, first build it using the Makefile.
Next, simply run the server as so:

./server -r /path/to/web/root -p 1234

Where \emph{-r} is the absolute path to the web root, where documents should be served out of (required), and \emph{-p} is the port to run the server on. If \emph{-p} is not specified, the server will be ran on its default port, 9529.

\section{Client/Server Output}

\end{document}